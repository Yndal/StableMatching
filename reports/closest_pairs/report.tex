\documentclass{tufte-handout}
\usepackage{amsmath}
\usepackage[utf8]{inputenc}
\usepackage{mathpazo}
\usepackage{booktabs}
\usepackage{microtype}

\pagestyle{empty}


\title{Closest Pair Report}
\author{Sigurt Dinesen, Gustav Røder \& Lars Yndal Sørensen}

\begin{document}
\maketitle

\section{Results}
Our implementation
produces the correct result, for all of the provided input/output examples
(\texttt{closest-pair-out.txt}). The calculated results, and given
example outputs, are compared when the program is run without parameters.

\section{Implementation}
The implementation follows the pseudo code from page 230 in \emph{Kleinberg and
Tardos, Algorithm Design Addison--Wesley 2005} closely.

For input sizes $n \leq 3$, we run a naive, pairwise $O(n^2)$ comparison algorithm.
For larger inputs, we recursively divide the plane into two parts, each with
$n/2 \pm 1$ points and find the closest point-pair in the left half, the right
half and the pairs who's connecting edges cross the partition -- returning the
closest of the three pairs.
The "crossing" pair is found using the trick from the book, taking into account
the minimum distance of the "left" and "right" pairs, allowing it to be done in
linear time.

The recursive function has a cutoff to the $O(n^2)$ solution, as mentioned
above.
It should be noted that $O(n^2)$, with the restriction that $n<=3$, is
considered to be constant time.

\subsection{Performance}
The running time is $n \log(n)$. This is illustrated in the following, by use of
the \texttt{master theorem}.

The running time is an instance of the equation
$$T(n) = a T\left(\frac{n}{b}\right) + f\left(n\right)$$
Which means we can analyse it using the \texttt{master theorem}:
As we recursively divide the input into two parts of $O(n/2)$ size, we have
$a=b=2$ in the above equation, which gives us
$$k = \log_b(a) = \log_2(2) = 1$$

At each level of recursion:
\begin{itemize}
	\item The four lists (named \texttt{Q\_x, Q\_y, R\_x, and R\_y} in both
		the book and our implementation) containing the points of each
		partition, in increasing order by the $x$ and $y$ coordinate
		respectively, are constructed at $O(n)$ cost.

	\item The set of points within $\delta$ distance of the partitioning
		line is found at $O(n)$ cost.

	\item Each point in that set is compared to the 15 following points, at
		$O(n*15) = O(n)$ cost.
\end{itemize}
In summary: $f\left(n\right) \in O(n)$

Returning to the \texttt{master theorem}, we observe that as $n = n^1 = n^k$:
$$f(n) \in O\left(n^k \log^0(n)\right)$$

which, according to the \texttt{master theorem}, means that the recursive
algorithm runs in
$$O(n^k \log^{0+1}(n)) = O(n \log(n))$$
Before the recursion, the input is ordered by both $x$ and $y$ coordinates,
giving $O(n \log(n))$ preprocessing cost, which leaves the total running time as
$$O\left(n \log(n)\right)$$
\end{document}
