\documentclass{tufte-handout}
\usepackage{amsmath}
\usepackage[utf8]{inputenc}
\usepackage{mathpazo}
\usepackage{booktabs}
\usepackage{microtype}

\pagestyle{empty}


\title{Interval partitioning report}
\author{Sigurt Dinesen, Gustav Røder \& Lars Yndal Sørensen}

\begin{document}
  \maketitle

  \section{Results}
  The implementation can parse input files, and produce the expected output
  compared to the existing output files.

  When the program isn't given any starting arguments, it will run all input
  files, and compare the resulting resource depth with that of the example
  outputs. The result is as follows:

  \begin{quotation}
 
	ip-1.in is ok? true\\
	ip-2.in is ok? true\\
	ip-3.in is ok? true\\
	ip-rand-100k.in is ok? true\\
	ip-rand-10k.in is ok? true\\
	ip-rand-1M.in is ok? true\\
	ip-rand-1k.in is ok? true\\

  \end{quotation}

  \section{Implementation details}
	The implementation is based on the pseudo code from page 166 in Kleinberg and Tardos, \emph{Algorithms Design}, Addison--Wesley 2005.%
	The content of the input file is parsed into Job objects, which are
	ordered by increasing start-time by use of the standard Java
	\texttt{java.util.PriorityQueue}
	Jobs are then assigned to the first available resource. This ordering is
	also maintained by use of the Java priority queue.

	let $n$ denote the number of jobs.
	The initial sorting of the jobs takes $O(n \log(n))$. We then traverse
	the jobs in sorted order, taking an additional $O(n \log(n))$ time, as
	we do one dequeue operation for each job, each operation taking
	$O(\log(n))$ time.
	For each job, we dequeue a resource from the resource queue, and add it
	again, taking $2 * O(\log(n))$ time for each $n$.
	Thus, the total running time comes to $4n \log(n) \in O(n \log(n))$

	Notably, the jobs do not need to be ordered in a priority queue, as we
	do not make insertions after the first ordering. Instead, a sorted list
	could have been used. A list would save us $O(n \log(n))$ operations,
	but would not change the running time in asymptotic (big-O) notation.

\end{document}
